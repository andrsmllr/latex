%
% paragraph_formatting
% see: http://en.wikibooks.org/wiki/LaTeX/Paragraph_Formatting
%

\documentclass[11pt,a4paper,titlepage,onecolumn]{article}

%
% The parskip package handles spacing around paragraphs in case of 0 indentation.
%
\usepackage{parskip}

%
% The alltt package supplies an environment similar to verbatim,
% but commands can be used within.
%
\usepackage{alltt}
\renewcommand{\ttdefault}{txtt} % Add this line to use \textbf{} within alltt env.

%
% The hyperref package helps handling URLs.
%
\usepackage{hyperref}
\hypersetup{pdfborder={0 0 0}} % This removes the (colored) box/frame around URLs.
%\urlstyle{same} % This will display URLs in the normal font used in the document.

%
% The moreverb package provides line numbers for a verbatim environment.
%
\usepackage{moreverb}

\begin{document}

%
% Left justify text.
%
\begin{flushleft}
    This text is left justified.
\end{flushleft}

% The command form is only for use in special environments like tables.
% {\raggedright{}More text on the left.}

%
% Right justify text.
%
\begin{flushright}
    This text is right justified.
\end{flushright}

% The command form is only for use in special environments like tables.
%{\raggedleft{}More text on the right.}

%
% Center text.
%
\begin{center}
    This text is center justified.
\end{center}

% The command form is only for use in special environments like tables.
%{\centering{}More text in the center.}

%
% Setting the size of a paragraph indent.
%
\setlength{\parindent}{1cm}

%
% Setting the size of vertical whitespace around a paragraph.
%
\setlength{\parskip}{1cm plus4mm minus3mm}

%
% Manual control of paragraph indentation.
%
\indent
\noindent

%
% Add additional indent to lines of a paragraph.
%
\hangindent=0.5cm % Additional indent to the left.
Lipsum
\leftskip=0.1cm
Lapsum
\hangafter=0.6cm % Additional indent to the right.
Lopsum
\rightskip=0.2cm
Lupsum
\parfillskip=0.7cm

%
% Paragraph line breaking.
%
\paragraph{Some paragraph} ~\\ % The ~ is to prevent "There's no line here to end" whining.

%
% Various ways to do manual line breaking. A ~ is added to avoid some TeX whining.
%
~\newline % Break the line.
~\\ % Same as \newline but shorter.
~\\* % Break the line but prohibit a page break thereafter.
~\linebreak[3] % Break line with given priority, 0=lowest...4=highest.
~\break % Break the line without filling it. TeX command.
~\hfill\break % Same as \break but fills the line before breaking it. Result as \newline.
~\par{New paragraph name} % Start a new paragraph, only within a paragraph(?). TeX command.

%
% verbatim environment.
%
\begin{verbatim}
This text will be output
as it is written, including
a l l    s p a c e s.
Commands are not recognized while in the verbatim environment.
Thus \newline is printed as typed.
The only command that is recognized is the one to end the verbatim environment.
\end{verbatim}

\verb|There is also a command version of the verbatim environment.|

%
% alltt environment.
%
\begin{alltt}
This text will also be output as it is written,
but opposed to the verbatim environment commands like emph
can be be used, \emph{like this}.
\textbf{Bolt text.}
\textit{Italic text.}
Formulas are also possible \(y = \sum \frac{1}{x}\).
\normalfont{}Returned to normal text while still in the alltt environment.
\end{alltt}

%
% hyperref.
%
With the hyperref package URLs like \url{http://en.wikibooks.org/wiki/LaTeX/Paragraph_Formatting} are displayed correctly.

%
% moreverb. Somehow broken? Let's show how the comment environment works then.
%
\begin{comment}
\begin{listing}
-- Buffer input signals.
process (clk) is
begin
	if rising_edge(clk) then
		data_i_d1 <= data_i;
	end if;
end process;
\end{listing}
\end{comment}

%
% Quoting.
%
\begin{quote}
This is a short quote with the quote environment.
\end{quote}

\begin{quotation}
This is a quote which is usually longer than just one single line of text,
in which case the quotation environment should be used.
\end{quotation}

\begin{verse}
The verse environment\\
is used\\
for poetry and other\\
things.\\
~\\
I don't know\\
at all\\
if this will be\\
useful\\
to me.
\end{verse}

\end{document}

